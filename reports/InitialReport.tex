\documentclass[IN,11pt,twoside,openright,english]{article}

\begin{document}


\title{%
  \Huge \textbf{Initial Report} \\
  \vspace{1.5cm} \large \textbf{Peer-to-Peer Systems and Security} \\
    \vspace{0.5cm} \textbf{Network Size Estimation} \vspace{1cm} }

\author{%
	\large Group 37 \\
	\large Dominik Winter \\ 
	\large Stefan Armbruster}
\date{}
\maketitle

\newpage

\section{Group Structure}

\begin{itemize}
\item \textbf{Teamnumber:} 37
\item \textbf{Teammembers:} Dominik Winter, Stefan Armbruster
\item \textbf{Subproject:} Network Size Estimation
\end{itemize}

\section{Operating System and Programming Language}

As we both are familiar with Linux we will utilize Ubuntu as OS for our project. In addition, one of us is using MacOS on his laptop. 
\newline
\newline
In addition we intend to choose Python 3.6 as programming language for this project, as it comes with many benefits for network related programs and has a lot of libraries that can help us in this specific field.

\section{Build System}

For build system, we will use \texttt{make}. With \texttt{make} we can automatically build executable programs and libraries from source code by reading files called \textit{Makefiles} which specify how to derive the target program.

\section{Quality Assurance}

A popular testing framework for Python programs is \texttt{PyUnit}. In addition, we will consider \texttt{PyTest} as testing framework for our project.

\section{Libraries}

As we cannot estimate required libraries for the project yet, we will update this section later during the project. However, libraries such as \texttt{pyp2p}, \texttt{pycrypto} and \texttt{twisted} might be helpful at a first glance.

\section{Licensing}

For licensing we choose the MIT license. As a permissive license, it puts only very limited restriction on reuse. It grants the freedom to modify and distribute the code as long as one includes the original copyright.

\section{Programming Experience}

Dominik has worked as a quality assurance engineer during his bachelor studies at the company \textit{genua gmbh} and is experienced in programming languages such as Ruby, Perl, Python and JavaScript. In addition, he wrote his bachelor thesis at the chair of Network Architectures and Services in the field of network function virtualization (NFV).\newline
\newline
Stefan has good or rather basic programming language skills in Java, Python, Matlab, JavaScript and Ocaml. He could gain fundamental experiences in the field of networks within the practical course Systemadministration and he was part of the network administration team of the student dormitory.
\section{Workload}
As both of us have basic coding experience in network related projects we will split the design and coding work equally. In addition, we will meet every week during different lectures, where we can manage project related tasks and futher steps.
\begin{itemize}
\item \textbf{Design/Logic:} Dominik (40\%), Stefan (60\%)
\item \textbf{Implementation} Dominik (50\%), Stefan (50\%)
\item \textbf{Documentation:} Dominik (70\%), Stefan (30\%)
\end{itemize}


\section{Issues}

No issues for the moment.

\end{document}
